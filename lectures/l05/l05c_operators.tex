\section{Custom Operators}

\begin{frame}
	\centering
	\Huge \textbf{Custom Operators}
\end{frame}

\begin{frame}[fragile]
  \frametitle{Custom Operators}
  \begin{columns}[T]
    \begin{column}{0.5\textwidth}
      \begin{itemize}
        \item Fortran allows you to define complex derived types.
        \item But using them directly for calculations leads to cumbersome code (and bugs, lots and lots of bugs)
      \end{itemize}
    \end{column}

    \begin{column}{0.5\textwidth}
    \lstinputlisting[firstline=1,lastline=6]{code.f90}
    \lstinputlisting[firstline=8,lastline=16]{code.f90}
    \end{column}
  \end{columns}
\end{frame}

\begin{frame}[fragile]
  \frametitle{Custom Operators}
   Code can be simplified (and made more robust) by using functions.
  \begin{columns}[T]
    \begin{column}{0.5\textwidth}
          \lstinputlisting[firstline=18,lastline=35]{code.f90}
    \end{column}

    \begin{column}{0.5\textwidth}
    \lstinputlisting[firstline=37,lastline=45]{code.f90}
    \end{column}
  \end{columns}
\end{frame}

\begin{frame}[fragile]
  \frametitle{Custom Operators}
   And defining custom operators can make your code even cleaner.
  \begin{columns}[T]
    \begin{column}{0.5\textwidth}
          \lstinputlisting[firstline=48,lastline=69,basicstyle=\ttfamily\tiny]{code.f90}
    \end{column}

    \begin{column}{0.5\textwidth}
    \lstinputlisting[firstline=75,lastline=83,basicstyle=\ttfamily\tiny]{code.f90}
    \end{column}
  \end{columns}
\end{frame}


\begin{frame}[fragile]
  \frametitle{How to Define a Custom Operator}
  Custom operators are defined inside a module using the \texttt{operator} keyword.
  \vspace*{2mm}

  \begin{columns}[T]
    \begin{column}{0.5\textwidth}
      \begin{itemize}
            \item Standard operators (+, -, *, /) \lstinputlisting[firstline=86,lastline=88]{code.f90}
      \end{itemize}
    \end{column}

    \begin{column}{0.5\textwidth}
      \begin{itemize}
            \item A new operator named \texttt{.op.} \lstinputlisting[firstline=90,lastline=92]{code.f90}
      \end{itemize}    \end{column}
  \end{columns}

  \vspace*{2mm}
  The functions implementing the operators are defined later in the module after the \texttt{contains} keyword.
\end{frame}
